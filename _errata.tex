\documentclass[a4paper]{article}

\usepackage[hangul]{kotex}
\usepackage{amsfonts, amssymb, amsmath}

\title{박유성, 김기환 2004, 기초금융통계와 연습문제 해설 오류 모음}
\author{2018년 2학기 CE730 금융과 통계 수강생들}

\begin{document}
\maketitle
\section{주텍스트 오류} % (fold)
\label{sec:주텍스트 오류}
\subsection{4장} % (fold)
\label{sub:주4장}
\paragraph{p.67} % (fold)
\label{par:p_67}
예제 4-2에서는 부인사망확률이 7\%라고 되어 있으나, 해설집에는 9\%로 되어 있어 정답이 서로 다르게 나옴. 유의하길 바람 (2018f 김인회)
% paragraph p_67 (end)
% subsection 4장 (end)
% section 주텍스트 오류 (end)

\section{연습문제 오류} % (fold)
\label{sec:연습문제 오류}
\subsection{4장} % (fold)
\label{sub:보조4장}
\paragraph{p.26} % (fold)
\label{par:p_26}
연습문제 4.2 A (1) "삼품" 이 아니라 "상품"임. (2018f 이예림)
% paragraph p_26 (end)
\paragraph{p.28} % (fold)
\label{par:p_28}
표준편차는 35가 아니라 3.5임 (2018f 이예림)
% paragraph p_28 (end)
\paragraph{p.32} % (fold)
\label{par:p_32}
4.10 해설에 $P(X\le 10,000)$이 맞음 (2018f 김인회)
% paragraph p_32 (end)
\paragraph{p.33} % (fold)
\label{par:p_33}
4.11 해설에 $_{200}C_{30}$ 이어야 함 (2018f 김인회)
% paragraph p_33 (end)

\paragraph{p.43} % (fold)
\label{par:p_43}
교재 기초금융통계의 106p 4.31- 5번 문제의 답이 답지에는 
5. $E(X)= 0*0.04+ 1*0.11+ … + 7*0.03 = 2.9196 $이라고 나와있지만
실제로 $X$의 기댓값을 구해보면,
$E(X)= 0*0.04+ 1*0.11+ 2*0.18+ 3*0.24+ 4*0.14+ 5*0.17+ 6*0.09+ 7*0.03$이므로 답은 $3.35$가 나옵니다.
이로 인해 6번 문제의 답에도 오류가 있는데, 답지에는
6. $Var(X)= 0^2*0.04+ 1^2*0.11 … 7^2*0.03 – (2.9196)^2 = 5.6659$ 이라고 나와있지만
실제로 $X$의 분산을 구해보면,
$0^2*0.04+ 1^2*0.11+ 2^2*0.18+ 3^2*0.24+ 4^2*0.14+ 5^2*0.17+ 6^2*0.09+ 7^2*0.03 – (3.35)^2= 2.9675$가 나옵니다. (2018f 김연희)
% paragraph p_43 (end)
% subsection 4장 (end)
% section 연습문제 오류 (end)
\end{document}